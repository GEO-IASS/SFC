\documentclass[a4paper, 12pt]{article}
\usepackage[left=1.5cm, text={18cm, 25cm}, top=2.5cm]{geometry}
\usepackage[utf8]{inputenc}
\usepackage[czech]{babel}
\usepackage{cite}
\usepackage{graphicx}
\usepackage{float}
\usepackage{amsmath}
\newcommand{\myuv}[1]{\quotedblbase #1\textquotedblleft}

\title{Optimalizační algoritmus ACO, technická dokumentace}
\author{Martin Hruška, xhrusk16@stud.fit.vutbr.cz}

\date{}
\begin{document}

\maketitle

\section{Úvod}
\label{sec:intro}
Ant colony optimization (ACO), neboli Optimalizace mravenčí kolonií je optimalizační algoritmus spadající do umělé inteligence,
konkrétně patří mezi metody inteligence hejna. Algoritmus hledá optimální řešení pomocí vzájemně komunikujících agentů (umělých mravenců), kteří
inkrementálně hledají nejlepší cestu grafem s~váženými hranami \cite{aco:main}. 
Z~tohoto důvodu musí být optimalizační problém popsatelný právě váženým grafem, tudíž se metoda používá především pro řešení diskrétních problémů.

Cílem této práce je vytvoření nástroje pro demonstraci činnosti ACO algoritmu, který umožní s~algoritmem experimentovat, zároveň bude jednoduše rozširitelný
o~další modifikace algoritmu, ale již v~základní verzi bude implementovat více verzí algoritmu. Implementačním jazykem je C++ a je použit objektově orientovaný
přístup pro zvýšení modifikovatelnosti a rozšiřitelnosti kódu. Nástroj má podobu aplikace s~rozhraním v~příkazové řádce.

V~tomto dokumetu bude popsán samotný algoritmus \ref{sec:algorithm}, dále architektura aplikace \ref{sec:design},
některé implementační detaily \ref{sec:implementation}, experimenty provedené s~algoritmem a jejich vyhodnocení \ref{sec:eval}.
Součástí dokumentu jsou dvě přílohy obsahující uživatelskou příručku \ref{app:help} a popis formátu vstupního grafu \ref{app:format}.
% neco o aco algoritmus, co je cilem teto prace - rozsiritelnost, lehka pouzitelnost a moznost experimentovani, osnovu

\section{ACO Algoritmus}
\label{sec:algorithm}
% obecny popis, dat nejakou intuici, pripadne obrazek
Jak bylo uvedeno v~úvodu \ref{sec:intro}, algoritmus pracuje s~množinou mravenců, kteří inkrementálně budují řešení daného optimalizačního problému tak, že
hledají optimální cestu ve váženém grafu popisujícím daný problém. Za řešení úlohy považujme cestu procházející všemi vrcholy grafu právě jednou s~tím, že se
mravenec nakonec cesty vrací do počátečního uzlu. Váhy hran grafu udávají vzdálenost vrcholů, které spojují. Každá hrana také má určitou hladinu feromonu, který
vypouští mravenci při každém průchodu. Existují různé varianty ACO algoritmu, nyní bude ovšem posána jeho základní verze nazvaná Ant System.

Řešení optimalizačního problému probíhá v~zadaném počtu iterací, při čemž v~každé iteraci vytvoří každý mravenec svoje řešení.
Počáteční uzel, ze kterého každý mravenec začiná vytvářet cestu je vybrán náhodně.
V~každém kroku tvorby řešení musí mravenec zvolit další hranu, po níž bude pokračovat do dalšího uzlu. Volba hrany záleží na vzdálenosti uzlů, které spojuje, a
hladině feromonu na dané hraně. Ty určují pravděpodobnost, se kterou si mravenec vybere danou hranu dle následujícího vzorce:
\begin{center}
  $p(c_{ij}|s_k^p) =
   \begin{cases} 
      \frac{\tau^{\alpha}_{ij}\eta^{\beta}_{ij}}{\sum\limits_{c_{il}\in N(s_k^p)}{\tau^\alpha_{il}\eta^\beta_{il}}} & j \in N(s_k^p) \\
      0 & otherwise 
   \end{cases}
   $
\end{center}
kde $c_{ij}$ je hrana z~uzlu $i$ do uzlu $j$, $s_k^p$ je částečné řešení mravence $k$, $N(s_k^p)$ je množina vrcholů nepatřících do $s_k^p$, $\eta_{ij}$ je 
dáno vztem ($1/d_{ij}$), $d_{ij}$ je vzdálenost mezi uzly $i$ a $j$, $\tau_{ij}$ je hladina feromonu na hraně $c_{ij}$, $\alpha$ je váha feromonu na hraně
a $\beta$ je váha vzdálenosti $i$ a $j$.

Jakmile jsou řešení pro dané mravence hotova, najde se mravenec s~nejkvalitnějším řešním v~dané iteraci a pokud je toto lepší, než aktuálně nejlepší řešení
z~doposud provedených iterací, tak je uloženo na jeho místo.

Hladina feromonu je aktualizována na konci každé iterace a to tak, že mravenec vypouští po cestě, kterou prošel grafem, množství feromonu přímo uměrné kvalitě
ním vytvořeného řešení. V~různých verzích algoritmu se technika aktualizace feromonu, množství feromonu vypouštěného mravencem, či počet mravenců vypouštějících 
feromon může různit. V~popisované verzi Ant System je feromon na hraně z~vrcholu $i$ a $j$ aktualizován následujícím způsobem:
\begin{center}
  $\tau_{ij}=(1-\rho)\tau_{ij}+\sum\limits_{k=1}^{n}\Delta\tau_{ij}^k$
\end{center}
kde $\rho \in <0,1>$ je evaporační koeficient a $\Delta\tau_{ij}^k$ je množství fermon přidaný na hranu z~$i$ do $j$ $k$-tým mravencem a je definováno následovně:
\begin{center}
  $\Delta\tau_{ij}^k = 
  \begin{cases}
    \frac{Q}{L_k} & \emph{pokud je hrana } ij \emph{ v~řešení k-tého mravence}\\
    0 & jinak
   \end{cases}
   $
\end{center}
kde $Q$ je konstanta udávající feromon vyloučený jedním mravence a $L_k$ je cena řešení (t.j. délka cesty) $k$-tého mravence.

Algoritmus je prováděn, dokud není dosažen počet iterací zadaný uživatelem.


\subsection{Modifikace}
\label{subsec:modif}
V předchozí částí byl popsán ACO algoritmus v jeho základní verzi -- Ant systemu. Existují ovšem modifikace tohoto přístupu, které spočívají především ve změně
akutalizace feromonu na hranách. Některé z modifikací budou popsány v následujícím části dokumentu.
% jake jsou rozsireni algoritmu
\section{Návrh}
\label{sec:design}
% rozhozeni do trid, core tridy
\subsection{Modularita}
\label{subsec:modularity}
% popsat moznost rozsireni, trade of obecnost/zbytecne parametry

\section{Implementace}
\label{sec:implementation}
% implementace

\section{Experimenty a vyhodnocení}
\label{sec:eval}
% popsat nejaky rozumny graf/nakreslit ho -> udelat graf vzadelnosti odchylek v zavislosti na poradi prochazky

\section{Závěr}
\label{sec:concl}
% zhodnit co bylo udelano - knihovna pro experimentovani, snadno rozsiritelna

\appendix
\section{Uživatelský manuál}
\label{app:help}
\section{Formát vstupního souboru}
\label{app:format}

\newpage
\bibliography{literatura}
\bibliographystyle{czplain}
\end{document}
